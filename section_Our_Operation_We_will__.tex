\section{Our Operation}

We will operate by maintaining a bare-bones yet sufficient infrastructure for scientific collaboration and social engagement. What is bare-bones yet sufficient has to be discussed with everyone and we will probably make adjustments on-the-fly, but the way I see it right now is: a platform for code sharing, a platform for writing papers, a platform for videoconferences, email and web server, expert review of scientific and outreach proposals, expert database management, expert social media management, expert fundraising and brand-building, expert networking and recruiting. We will need money to maintain a small server, pay for ad-hoc subscriptions to scientific journals, pay for the publication of open access articles, pay for the application for patents (that will be free to use for noncommercial purposes), pay for ad-hoc code licenses and maintenance, pay for miscellaneous outreach costs, necessary travel, start saving so that we can pay people even when they don't have grants, organization of a physical annual meeting, etc. 

People admitted to be part of the Institute have to design their own position. This includes communicating to a committee created by the Board of Directors what their research plans are, how the can contribute to the transdisciplinary research mission of the Institute and how they want to be evaluated after 5 years (metrics of success) which could include writing and maintaining scientific code, outreach, writing authoritative articles on technology and social trends (being a futurist), etc. At the beginning we will probably accept anyone who has a good idea for a research project/grant proposal and shares our values. The officers would also hire database and computer experts, artists and other people needed, at the beginning probably as contractors. The Institute will never be big enough that members don't know each other personally, and this will cap its size to about 15-20 people.

The vision that we have for the future is that the current university-centered research model will be dramatically and fundamentally changed by the proliferation of computers that has changed almost every other area (academia tends to move slowly and usually reluctantly). We believe that other research institutes with the same business model will arise and much of the science in the future will be done in the way we describe above. Of course, not all institutes will share our values of transdisciplinary science, social engagement and outreach. Others might focus on putting together a group of experts in a particular field, focus on starting new business ventures, be for-profit corporations, etc. Nevertheless, we think that we gain a lot from being the first (to the best of our knowledge) in this area and we will try to capitalize on it.
