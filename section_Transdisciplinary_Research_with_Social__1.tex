\section{Transdisciplinary Research with Social Impact}

We happen to believe that science and technology should be useful to society and that we have a responsibility to educate society about science and technology. We will be radically close to society by being aggressive in social media, by reaching people in nontraditional ways such as reddit AMAs, by engaging in nontraditional (for science anyways) fundraising such as kickstarter.com and by making science and technology outreach a hallmark of our institution. We want to attract socially mindful scientist that would want to collaborate with local (for that particular scientist) schools and universities. If the scientist wants, outreach could be considered for his or her review. Everyone in the institute will be strongly encouraged to adhere to open science principles (https://en.wikipedia.org/wiki/Open\_science) which include: open data, open source, open methodology, open peer-review, open access and open educational resources. We will have members in the institute, everyone who contributes a little bit of money (like \$5) or volunteers time. Perks for them could be, for example, access to members of the institute during online "office hours," etc. 

We also happen to believe that science and technology outreach are avenues to reach underserved populations and that science and technology careers are a way out of poverty. We will make an impact on society and on the lives of other people by reaching out to underserved populations in general, and one-at-a-time by identifying motivated and intellectually promising individuals and by providing them both opportunities for growth and visibility. Even with the proliferation of information, it seems like underserved students just do not seriously think about the possibility of a successful career in science or engineering. We will achieve this in part by operating a permanent science and tech incubator for mixed groups of high school and college students that will have access to an expert, providing hands-on experience, opportunities for networking and letters of recommendation. Art is another avenue to reach people, and "technical art" will play an increasingly important role in the future, as more and more complex data needs to be visualized and communicated. This new role will be added to the traditional science, technology, engineering and mathematics aggregation. STEM will become STEAM.

A third belief is that the proliferation of big data in every scientific, industrial and social branch is an opportunity for transdisciplinary collaboration. In 1960, Eugene Wigner wrote "The Unreasonable Effectiveness of Mathematics in the Natural Sciences" that the mathematical structure of a physical theory is predictive, even though there does not seem to be any a priori reasons for this to be the case. Once you have an equation, it is possible to predict phenomena not investigated in the original experiment. Complex systems with emergent properties are more difficult to describe mathematically, but oftentimes holistic models (usually in the form of algorithms) explain these properties. We believe that models developed for one emergent system might explain properties of other emergent systems. You can call it "The Unreasonable Effectiveness of Algorithmic Models in All Sciences." We propose to test this hypothesis by putting together a team that includes collaborators from fields that could potentially be investigated using these models, for example analytical jurisprudence, analytical philosophy and linguistics. 

A final belief is that with big power comes big responsibility. Accelerated progress in the field of artificial intelligence, Internet of Things, and others is moving humans into completely uncharted territory. The technological and scientific tools and capabilities accessible to our civilization could be used for good or for evil or have unintended consequences. We will advocate for the ethical use of these tools and capabilities and will continually inform society of trends, risks and opportunities that we observe in these areas.
