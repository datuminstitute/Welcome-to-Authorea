\section{Market Need and Opportunity for a New Scientific Workflow}
We think there is a market for a research institute that is completely online and that functionally would be like banks that operate completely online, hence have lower costs that brick-and-mortar banks, and can give the clients a better interest rate. In this case the "lower interest rate" would be a much lower overhead (say, 10\% instead of 62\% for Caltech, 55\% for MIT, etc.) 

Another analogy is that the institute would also work like ARM which is a fabless semiconductor company, they design semiconductors (notably for the iPhone) and outsource the manufacturing. We will establish research collaborations with groups at universities and national laboratories that have equipment to do experimental work and will apply for time at user facilities at national laboratories, which is free if awarded. The Institute will be focused on transdisciplinary work and by not maintaining physical laboratories we will incentivize collaboration with other institutions and will allows to continue to maintain costs low. 

We will have a strong focus on computational and simulational work, and we will outsource all our computations to the cloud. This will take advantage of the economies of scale. We will try to develop a strong relationship with Google, Amazon and Microsoft to obtain better prices on the computation, storage, and data transmissions and we will use open-source software as much as possible. Out research institute will be much closer to industry than most academic institutions currently are and we will adopt many of its operating principles such as agile. 

We believe that this market exists because the three main problems faced by early career scientists in computationally intensive areas (and that is increasingly more and more areas). These three problems are: low salaries compared to industry; low stability as the scientist usually moves a few times before finding something permanent and faces the two-body problem or the living-in-a-terrible-place problem; and advancement problems for scientists who want to write better, more robust code and publicly available code for reproducible science, but don't have the time due to the publish-or-perish state of affairs. See https://jakevdp.github.io/blog/2014/08/22/hacking-academia/. These problems discourage many from continuing in academic research. We believe, that right away, we can offer salaries that are more competitive with industry by having a much lower overhead and we alleviate the low stability problem by being online. As the institute matures and gets financially better established, we will support scientists permanently so that they are not on soft money (we don't have funding, people will have to be on soft-money initially). We will not have tenure, but scientists will be evaluated every 5 years and the review will take into consideration software shared with the scientific community, among other things, and not necessarily only peer-reviewed publications. This will solve the low stability and the career progression problems.