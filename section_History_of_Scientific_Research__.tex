\section{History of Scientific Research as an Institution}
Brief historical introduction to the origins of the research institute as an organizational arrangement and its relationship to the scientific endeavor. Mention first learned societies in England and France, brief history of the journal as the channel of communication, the article as the unit of research. Mention the ad-hoc connection between teaching at the university level and research at a university. Mention other models of scientific research such as the national laboratory and research in industry. Identify how previous revolutions, e.g., industrial, have revolutionized science 

In an environment with very limited resources it made sense to have a system in which "experts" decided what research directions or research results were more important and hence more deserving of funding or publication. There will always be a place for experts and elites in our society, but we are no longer limited by what it is possible to fit on a piece a paper. We will follow the open access model for publications: publish now, judge later. Arthur Clarke said that: "If an elderly but distinguished scientist says that something is possible, he is almost certainly right; but if he says that it is impossible, he is very probably wrong." The role of experts should not be to strangle science by recommending that an article is not published. Instead, it should be to tell the public which results are noteworthy once they have been published.

Every paper published by members of the Institute will be accompanied by a press release that will explain to journalists and to the general public why the results are significant and how they relate with what we already know. In addition, members of the Institute will be encouraged to write about work that they consider particularly important in their field with the goal of explaining it to journalists and the public.

We recognize that statistics and metrics are a guide to but not a replacement for human insight and intelligence.