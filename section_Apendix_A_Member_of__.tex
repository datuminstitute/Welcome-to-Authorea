\section{Apendix}

A Member of the Board of Directors is not officially affiliated with the Institute but has a work relation. His or her responsibility boils down to give his/her opinion (if the member wants to do it) on issues pertaining the Institute, evaluate the performance, fire, or hire the Chief Executive Officer, and review its finances. The Chairman of the Board can invite whomever he or she deems qualified to be part of the Board and can take recommendations from current Members of the Board. Currently there are no term limits but the Board might change this at a later time. We will not pay our Members of the Board. The Board has the authority to form committees to address particular issues and members of these committees can be persons not affiliated with the Institute or who work for or with the Institute. A committee can be formed at the request of any Member of the Board or the CEO and the person requesting the formation of the committee might make non-binding recommendations on its composition. We will likely have permanent committees, for example to review grant applications.

The Institute will maintain work relations with individuals that can help to advance its mission. Such positions include that of Technical Expert, Industry Leader and Role Model. There is no affiliation requisite for these positions. They help us with our incubator, going to local high schools and universities or in any other way finding promising, underserved students who either have and idea for the incubator or could get behind an idea. The project in the incubator should be something that would benefit society in general or a sector of society. We will not pay people helping us in this capacity but we can pay for outreach or operating costs.

The Institute will have permanent scientists and other thought leaders officially and permanently affiliated to the Institute. Grant-seeking will occur through the Institue and the scientist will stipulate his or her salary in the grant proposal. 