\section{Articles of Incorporation}

FIRST. The name of this corporation shall be The Datum Institute.

SECOND. The place where the principal office of the corporation is to be located is the City of Portland, Multnomah County.

THIRD. The corporation is organized exclusively for charitable, educational, literary and scientific purposes under section 501(c)(3) of the Internal Revenue Code, or corresponding section of any future federal tax code. 

FOURTH. The names, residence addresses and number of the Directors of said corporation at the time of its incorporation are:

\begin{enumerate}
\item Jorge Alberto Munoz, Jr, Beaverton, Oregon
\item Jorge Alberto Lopez, El Paso, Texas
\item Tomas Hernandez, Portland, Oregon
\item Arturo Munoz, Houston, Texas
\item William Ford, San Francisco, California
\end{enumerate}

The authorized number of the Directors, which shall be not less than three, may be fixed or changed by a bylaw.  No such bylaw may be adopted, amended or repealed without the vote or written assent of members of said corporation entitled to exercise a majority of the voting power, or the vote of a majority of a quorum at a meeting of members duly called pursuant to the articles of incorporation or bylaw.

FIFTH. No part of the net earnings of the corporation shall inure to the benefit of, or be distributable to its members, directors, officers, or other private persons, except that the corporation shall be authorized and empowered to pay reasonable compensation for services rendered and to make payments and distributions in furtherance of the purposes set forth in the purpose clause hereof. No substantial part of the activities of the corporation shall be the carrying on of propaganda, or otherwise attempting to influence legislation, and the corporation shall not participate in, or intervene in (including the publishing or distribution of statements) any political campaign on behalf of or in opposition to any candidate for public office. Notwithstanding any other provision of these articles, the corporation shall not carry on any other activities not permitted to be carried on (a) by a corporation exempt from federal income tax under section 501(c)(3) of the Internal Revenue Code, or the corresponding section of any future federal tax code, or (b) by a corporation, contributions to which are deductible under section 170(c)(2) of the Internal Revenue Code, or the corresponding section of any future federal tax code.

SIXTH. Upon the dissolution of the corporation, assets shall be distributed for one or more exempt purposes within the meaning of section 501(c)(3) of the Internal Revenue Code, or the corresponding section of any future federal tax code, or shall be distributed to the federal government, or to a state or local government, for a public purpose. Any such assets not so disposed of shall be disposed of by a Court of Competent Jurisdiction of the county in which the principal office of the corporation is then located, exclusively for such purposes or to such organization or organizations, as said Court shall determine, which are organized and operated exclusively for such purposes.